\documentclass[a4paper, 12pt]{article}

% encoding, font, language

\usepackage[T1]{fontenc}
\usepackage[utf8]{inputenc}
%\usepackage[latin1]{inputenc}
\usepackage{lmodern}
\usepackage[ngerman, english, french]{babel}

\usepackage{nicefrac}

\usepackage[
    handwritten,
    nowarnings,
]{xcookybooky}

%\usepackage{blindtext}    % only needed for generating test text

\DeclareRobustCommand{\textcelcius}{\ensuremath{^{\circ}\mathrm{C}}}


\setcounter{secnumdepth}{1}
\renewcommand*{\recipesection}[2][]
{%
    \subsection[#1]{#2}
}
\renewcommand{\subsectionmark}[1]
{% no implementation to display the section name instead
}


\usepackage{hyperref}    % must be the last package
\hypersetup{%
    pdfauthor            = {Pierr Fierz},
    pdftitle             = {Une recette de Pierre},
    pdfsubject           = {Recette de cuisine},
    pdfkeywords          = {recette, xcookybooky},
    pdfstartview         = {FitV},
    pdfview              = {FitH},
    pdfpagemode          = {UseNone}, % Options; UseNone, UseOutlines
    bookmarksopen        = {true},
    pdfpagetransition    = {Glitter},
    colorlinks           = {true},
    linkcolor            = {black},
    urlcolor             = {blue},
    citecolor            = {black},
    filecolor            = {black},
}

\hbadness=10000	% Ignore underfull boxes
\setlength{\headheight}{20pt}

\begin{document}

%\include{tex/soupe_petit_pois}
\begin{otherlanguage}{french}

\setHeadlines
{% translation
    inghead = Ingrédients,
    prephead = Préparation,
    hinthead = conseil,
    continuationhead = suite,
    continuationfoot = suite à la page suivante,
    portionvalue = personnes
}

 
\begin{recipe}
[ % Optionale Eingaben
    preparationtime = {\unit[1]{h}},
    portion = \portion{4},
    %source = R. Gaus
]
{Crème froide de petits pois à la menthe}
    
    \graph
        {% Bilder
          small=creme-pois, % großes (längeres) Bild
          big=creme-pois-ingredients , % kleines Bil
        }
    
    \ingredients
        {% Zutaten
         \unit[300]{g} & de petits pois frais écossés\\
         \unit[5]{dl} & de bouillon de volaille froid\\
          \unit[1]{dl} & d'huile d'olive \\
          \unit[5]{cl} & de crème liquide \\
          30-50 & feuilles de menthe poivrée \\
          & sel, poivre \\
    }
    
    \preparation
        { % Zubereitung
      %\step Ecosser les petits pois. 
      \step Porter 2 litre d'eau à ébullition avec 20g de sel.
      \step Ajouter les petits pois frais, et compter 3 minutes de cuisson
      \step Rafraichir les petits pois dans de l'eau bien froide (avec des cube de glace). Dés qu'il sont bien froids, les égouter.
      \step Mettre les petits pois dans le bol d'un mixeur. Ajouter 30-50 feuilles de menthe (selon votre goût) puis mixer. 
      Verser ensuite \unit[1]{dl} d'huile d'olive, saler, poivrer, ajouter \unit[5]{cl} de créme liquide et mixer.
      \step Passer la crème dans un chinois ou une passoire très fine, puis la reserver au réfrigerateur.
      \step Servir la crème bien froide dans des assiettes individuelles, garnir avec 2 à 3 feuilles de menthe et d'un filet 
      d'huile d'olive.
    }
    
    \hint
    {% Tipp
        \unit[1]{kg} de petits pois frais donnent environs \unit[300]{g} de petits pois écossés.
    }

\end{recipe}

\end{otherlanguage}


\end{document} 
