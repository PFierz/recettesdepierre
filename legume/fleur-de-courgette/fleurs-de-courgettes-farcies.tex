\documentclass[a4paper, 12pt]{article}

% encoding, font, language

\usepackage[T1]{fontenc}
\usepackage[utf8]{inputenc}
%\usepackage[latin1]{inputenc}
\usepackage{lmodern}
\usepackage[ngerman, english, french]{babel}

\usepackage{nicefrac}

\usepackage[
    handwritten,
    nowarnings,
]{xcookybooky}

%\usepackage{blindtext}    % only needed for generating test text

\DeclareRobustCommand{\textcelcius}{\ensuremath{^{\circ}\mathrm{C}}}


\setcounter{secnumdepth}{1}
\renewcommand*{\recipesection}[2][]
{%
    \subsection[#1]{#2}
}
\renewcommand{\subsectionmark}[1]
{% no implementation to display the section name instead
}


\usepackage{hyperref}    % must be the last package
\hypersetup{%
    pdfauthor            = {Pierr Fierz},
    pdftitle             = {Une recette de Pierre},
    pdfsubject           = {Recette de cuisine},
    pdfkeywords          = {recette, xcookybooky},
    pdfstartview         = {FitV},
    pdfview              = {FitH},
    pdfpagemode          = {UseNone}, % Options; UseNone, UseOutlines
    bookmarksopen        = {true},
    pdfpagetransition    = {Glitter},
    colorlinks           = {true},
    linkcolor            = {black},
    urlcolor             = {blue},
    citecolor            = {black},
    filecolor            = {black},
}

\hbadness=10000	% Ignore underfull boxes
\setlength{\headheight}{20pt}

\begin{document}

\begin{otherlanguage}{french}

\setHeadlines
{% translation
    inghead = Ingrédients,
    prephead = Préparation,
    hinthead = conseil,
    continuationhead = ,
    continuationfoot = ,
    portionvalue = personnes
}


\begin{recipe}
[ % Optionale Eingaben
    preparationtime = {\unit[45]{m}},
    portion = \portion{4},
    bakingtime={\unit[15]{m}},
    bakingtemperature={\protect\bakingtemperature{
        topbottomheat=\unit[180]{\textcelcius}}},
    %source = R. Gaus
]
{Fleurs de courgettes farcies}
    
    \graph
        {% Bilder
          small=fleur-de-courgette, % großes (längeres) Bild
          big=fleur-de-courgette-ingredients , % kleines Bil
        }
    
    \ingredients
        {% Zutaten
         12 & fleurs de courgettes femelles (avec la petite courgette) \\
         \unit[100]{g} & de fromage de chèvre frai\\
         \unit[50]{g} & de mascarpone\\
         1 & grosse échalotes\\
         3 & cuillères à soupe d'herbes ciselé (persil plat, basilic, ...)\\
         \unit[1] & cuillères à café de piment d'Espelette\\
                  & huile d'olive\\
                  & beurre\\
                  & sel et poivre    
      }
    
    \preparation { % Zubereitung
      \step Dans une poêle faire revenir dans du beurre l'échalote finement ciselée durant 2 à 3 minutes. Saler.
      \step Mettre le fromage de chèvre dans un récipient et l'écraser à l'aide d'une fourchette.
      \step Ajouter le mascarpone et bien mélanger toujours à l'aide de la fourchette.
      \step Incorporer ensuite les échalotes, les herbes et le piment d'Espelette. Vérifier l'assaisonnement.
            Voilà la farce et prête.
      \step Frotter les petites courgettes avec un papier pour enlever les petits poils et couper le pédoncule. Ouvrir 
            délicatement la fleur et retirer le pistil.
      \step Avec une petite cuillère (ou une poche à douille) farcir les fleurs avec la préparation de fromage de chèvre.
            Bien refermer les pétales de la fleur.
      \step Badigeonner un plat allant au four avec de l'huile d'olive et y déposer les fleurs.
      \step Arroser les fleurs avec un filet d'huile d'olive et enfourner à \unit[180]{\textcelcius} durant \unit[15]{m} environ.  
    }
    
    \suggestion[Variantes] {
      Si l'on n'aime pas le fromage de chèvre, on peut le remplacer par de la ricotta
    }
    
    \hint {% Tipp
      \begin{itemize}
       \item Les fleurs de courgettes peuvent être utilisées pour une assiette de légumes (voir la photo) ou servie avec une viande
             ou une volaille.
       \item S'il reste de la farce, on peut l'utiliser pour confectionner des canapés. C'est succulent!
      \end{itemize}

      
    }

\end{recipe}

\end{otherlanguage}


\end{document} 
