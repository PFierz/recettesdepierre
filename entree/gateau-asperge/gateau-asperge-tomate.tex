\documentclass[a4paper, 12pt]{article}

% encoding, font, language

\usepackage[T1]{fontenc}
\usepackage[utf8]{inputenc}
%\usepackage[latin1]{inputenc}
\usepackage{lmodern}
\usepackage[ngerman, english, french]{babel}

\usepackage{nicefrac}

\usepackage[
    handwritten,
    nowarnings,
]{xcookybooky}

%\usepackage{blindtext}    % only needed for generating test text

\DeclareRobustCommand{\textcelcius}{\ensuremath{^{\circ}\mathrm{C}}}


\setcounter{secnumdepth}{1}
\renewcommand*{\recipesection}[2][]
{%
    \subsection[#1]{#2}
}
\renewcommand{\subsectionmark}[1]
{% no implementation to display the section name instead
}


\usepackage{hyperref}    % must be the last package
\hypersetup{%
    pdfauthor            = {Pierr Fierz},
    pdftitle             = {Une recette de Pierre},
    pdfsubject           = {Recette de cuisine},
    pdfkeywords          = {recette, xcookybooky},
    pdfstartview         = {FitV},
    pdfview              = {FitH},
    pdfpagemode          = {UseNone}, % Options; UseNone, UseOutlines
    bookmarksopen        = {true},
    pdfpagetransition    = {Glitter},
    colorlinks           = {true},
    linkcolor            = {black},
    urlcolor             = {blue},
    citecolor            = {black},
    filecolor            = {black},
}

\hbadness=10000	% Ignore underfull boxes
\setlength{\headheight}{20pt}

\begin{document}

\begin{otherlanguage}{french}

\setHeadlines
{% translation
    inghead = Ingrédients,
    prephead = Préparation,
    hinthead = conseil,
    continuationhead = ,
    continuationfoot = ,
    portionvalue = personnes
}

\begin{recipe}
[ % Optionale Eingaben
    preparationtime = {\unit[1]{h}},
    portion = \portion{4},
    bakingtime={\unit[30]{m}},
    bakingtemperature={\protect\bakingtemperature{
        topbottomheat=\unit[180]{\textcelcius}}},
    %source = R. Gaus
]
{Gâteau d'asperges, tomates confites et sauce hollandaise}

    \graph
        {% Bilder
          small=gateau-asperge, % großes (längeres) Bild
          big=gateau-asperge-ingredients , % kleines Bil
        }
        
    \ingredients
        {% Zutaten
         \unit[800]{g} & d'asperges blanche trés fraiche\\
         4 & blancs d'oeufs\\
         \unit[1]{dl} & d'huile d'olive\\
             & sel, piment d'Espelette \\      
      }
    
    \preparation { % Zubereitung
      \step Couper le bout des asperges, les éplucher et les nouer en deux bottes de grandeur égale (voire la photo)
            Cela permet une cuisson uniforme des asperges.
      \step Dans une grande casserole faire chauffer \unit[3]{l} d'eau avec \unit[30]{gr} de sel.
      \step Quand l'eau bout y plonger les asperges et laisser cuire à gros bouillons durant cinq minutes. 
            Retirer la casserole du feu et la couvrir. Après 5 à 10 minutes les asperge descende au fond de la
            casserole elles sont cuites à point. Les laisser refroidir sur du papier absorbant. Quand les asperges sont froides
            les couper en tronçon de 1 centimètre.
      \step Mettre les bouts d'asperge dans le bol d'un mixeur. Ajouter les quatre blancs d'oeufs et mixer pendant 1 minute.
      \step Ajouter l'huile d'olive, une cuillère à café de piment d'Espelette du sel et mixer pendant 1 minute. Vérifier 
            l'assaisonnement.
      \step Bien beurrer 4 à 5 ramequins individuels et les remplir avec la préparation.
      \step Pour la cuisson, mettre les ramequins dans un plat en pyrex puis verser de l'eau bouillante jusqu'aux deux tiers de la 
            hauteur des ramequins. Enfourner pendant \unit[30]{m} à \unit[180]{\textcelcius}.
      \step Démouler les gâteaux d'asperge sur du papier absorbant et les placer ensuite au centre d'une assiette. Disposer
            autour du gâteau quatre pétales de tomate et quatre cuillères à soupe de sauce hollandaise.
    }
    
    \suggestion[Sugestion] {
      Si l'on en trouve, parsemer l'assiette de cerfeuil.
    }
    
\end{recipe}



\begin{recipe}
[ % Optionale Eingaben
    preparationtime = {\unit[0.5]{h}},
    bakingtime={\unit[2.5]{h}},
    bakingtemperature={\protect\bakingtemperature{
        fanoven=\unit[90-120]{\textcelcius}}},
    portion = 16 pétales de tomates,
    %source = R. Gaus
]
{Tomates confites}
        
    \ingredients
        {% Zutaten       
         4 & tomates bien mures mais fermes\\
         2 & gousses d'ail\\
         2-3 & cuillères à soupe d'huile d'olive\\
           & quelques brins de thym\\
           & sel, poivre\\      
      }
    
    \preparation { % Zubereitung
      \step Enlever le pédoncule des tomates et faire une incision en forme de croix au dos des tomates. Tremper les 40 à
            \unit[50]{s} dans de l'eau bouillante puis dans de l'eau froide. Peler les tomates. Les couper en quart et
            retirer tous les pépin.
      \step Mettre une feuille de protection sur la plaque du four. Badigeonner les pétales de tomates d'huile et les déposer 
            (côté bombé) sur la plaque.
      \step Saler et poivrer puis déposer dans chaque tomate une fine tranche d'ail et un brin de thym.
      \step Enfourner pendant \unit[1]{h} à \unit[120]{\textcelcius} puis ensuite pendant 1.5 h à \unit[90]{\textcelcius}.
            Four ventilé de préférence.
      \step Retirer l'ail et le thym et laisser refroidir les tomates à température ambiante.
    }
\end{recipe}


\begin{recipe}
[ % Optionale Eingaben
    preparationtime = {\unit[10]{m}},
    portion = \portion{4},
    %source = R. Gaus
]
{Sauce hollandaise}
        
    \ingredients
        {% Zutaten       
	  2 & jaunes d'oeufs\\
	  \unit[140]{g} & de beurre\\
	  1/2 & citron\\
	      & sel, poivre\\     
      }
    
    \preparation { % Zubereitung
      \step Couper le beurre en petits cubes.
      \step Presser le 1/2 citron.
      \step Dans un récipient résistant à la chaleur battre les jaunes d'oeufs avec 2 cuillerées à soupe d'eau froide.
            Mettre au bain-marie chaud. Fouettez jusqu'à l'obtention d'une crème mousseuse et légère.
      \step Ajouter le beurre en plusieurs fois tout en continuant de fouetter.
      \step Saler, poivrer et incorporer le jus du citron avant de servir.
    }
    
    \hint{%
      \begin{itemize}
       \item Les tomates confites peuvent aussi être utilisées dans une salade.
       \item La sauce hollandaise peut ètre servie avec du poisson. Elle accompagne aussi très bien
	     les volailles.
      \end{itemize}

    }
\end{recipe}


\end{otherlanguage}


\end{document} 
