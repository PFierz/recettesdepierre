\documentclass[a4paper, 12pt]{article}

% encoding, font, language

\usepackage[T1]{fontenc}
\usepackage[utf8]{inputenc}
%\usepackage[latin1]{inputenc}
\usepackage{lmodern}
\usepackage[ngerman, english, french]{babel}

\usepackage{nicefrac}

\usepackage[
    handwritten,
    nowarnings,
]{xcookybooky}

%\usepackage{blindtext}    % only needed for generating test text

\DeclareRobustCommand{\textcelcius}{\ensuremath{^{\circ}\mathrm{C}}}


\setcounter{secnumdepth}{1}
\renewcommand*{\recipesection}[2][]
{%
    \subsection[#1]{#2}
}
\renewcommand{\subsectionmark}[1]
{% no implementation to display the section name instead
}


\usepackage{hyperref}    % must be the last package
\hypersetup{%
    pdfauthor            = {Pierr Fierz},
    pdftitle             = {Une recette de Pierre},
    pdfsubject           = {Recette de cuisine},
    pdfkeywords          = {recette, xcookybooky},
    pdfstartview         = {FitV},
    pdfview              = {FitH},
    pdfpagemode          = {UseNone}, % Options; UseNone, UseOutlines
    bookmarksopen        = {true},
    pdfpagetransition    = {Glitter},
    colorlinks           = {true},
    linkcolor            = {black},
    urlcolor             = {blue},
    citecolor            = {black},
    filecolor            = {black},
}

\hbadness=10000	% Ignore underfull boxes
\setlength{\headheight}{20pt}

\begin{document}

\begin{otherlanguage}{french}

\setHeadlines
{% translation
    inghead = Ingrédients,
    prephead = Préparation,
    hinthead = conseil,
    continuationhead = ,
    continuationfoot = ,
    portionvalue = personnes
}

 
\begin{recipe}
[ % Optionale Eingaben
    preparationtime = {\unit[1]{h}},
    portion = \portion{6-8},
    %source = R. Gaus
]
{Ragoût d'asperges et morilles}
    
    \graph
        {% Bilder
          small=ragout-asperge, % großes (längeres) Bild
          big=ragout-asperge-ingredients , % kleines Bil
        }
    
    \ingredients
        {% Zutaten
         \unit[800]{g} & d'asperges blanche trés fraiche\\
         \unit[200]{g} & de morilles fraiches\\
         2 & échalottes moyennes\\
         \unit[1]{dl} & de vin blanc sec\\
         \unit[1]{dl} & de fond de veau\\
         \unit[1]{dl} & de crème liquide\\
         \unit[7]{cl} & de vin jaune (du Jura)\\
         \unit[20]{g} & de beurre mannié\\ \\
                      & persil plat, thym \\
                      & sel, poivre, piment d'espelette \\      
      }
    
    \preparation { % Zubereitung
      \step Pour le beurre manié prendre \unit[15]{gr} de beurre coupé en petits morceaux. Saupoudrer de \unit[15]{gr} de
            farine et triturer avec une fourchette afin d'obtenir une pâte homogène.
      \step Couper le bout des asperges, les éplucher et les nouer en deux bottes de grandeur égale (voire la photo)
            Cela permet une cuisson uniforme des asperges.
      \step Dans une grande casserole faire chauffer \unit[3]{l} d'eau avec \unit[30]{gr} de sel.
      \step Quand l'eau bout y plonger les asperges et laisser cuire à gros bouillons durant cinq minutes. 
            Retirer la casserole du feu et la couvrir. Après 5 à 10 minutes les asperge descende au fond de la
            casserole elles sont cuites à point. Les laisser refroidir sur du papier absorbant. Quand les asperges sont froides
            les couper en tronçon de 1 centimètre.
      \step Ciseler finement les échalotes. Laver bien les morilles et les couper en rondelles d'un demi-centimètre.
      \step Faire chauffer un filet d'huile d'olive dans une poêle et y ajouter les échalotes. Saler et laisser revenir
            2 minutes. Ajouter ensuite les rondelles de morilles et laisser revenir 2 à 3 minutes.
      \step Ajouter alors le vin et le fonds de veau et laisser réduire à petit feu durant cinq minutes.
      \step Ajouter alors la crème et laisser mijoter durant cinq autres minutes.
      \step Incorporer le beurre manié puis tout à la fin le vin jaune. Ne plus faire cuire.
      \step Parsemer le plat de persil et de thym ciselé. Servir aussitôt.
      
    }
    
    \suggestion[Variantes] {
      \begin{itemize}
       \item Si l'on ne trouve pas de morilles fraiches prendre \unit[20]{gr} de morilles séchées.
       \item Si l'on n'a pas de vin jaune on peut utiliser du Xeres sec. Ce sera aussi moins cher.
      \end{itemize}
    }
    
    \hint {% Tipp
	 Les asperges sont fraiches, si elle grince lorsqu'on les frotte les une aux autres.
    }

\end{recipe}

\end{otherlanguage}


\end{document} 
